\documentclass[12pt]{article}
\usepackage{pmmeta}
\pmcanonicalname{DividedDifferenceInterpolationFormula}
\pmcreated{2013-03-22 16:19:13}
\pmmodified{2013-03-22 16:19:13}
\pmowner{CWoo}{3771}
\pmmodifier{CWoo}{3771}
\pmtitle{divided difference interpolation formula}
\pmrecord{8}{38447}
\pmprivacy{1}
\pmauthor{CWoo}{3771}
\pmtype{Theorem}
\pmcomment{trigger rebuild}
\pmclassification{msc}{39A70}

% this is the default PlanetMath preamble.  as your knowledge
% of TeX increases, you will probably want to edit this, but
% it should be fine as is for beginners.

% almost certainly you want these
\usepackage{amssymb}
\usepackage{amsmath}
\usepackage{amsfonts}

% used for TeXing text within eps files
%\usepackage{psfrag}
% need this for including graphics (\includegraphics)
%\usepackage{graphicx}
% for neatly defining theorems and propositions
%\usepackage{amsthm}
% making logically defined graphics
%%%\usepackage{xypic}

% there are many more packages, add them here as you need them

% define commands here

\begin{document}
Newton's \emph{divided difference interpolation formula} is the analogue 
of the Gregory-Newton and Taylor series for divided differences.

If $f$ is a real function and $x_0, x_1, \ldots$ is a sequence of distinct
real numbers, then we have, for any integer $n > 0$,
 \[ f (x) = f(x_0) + (x - x_0) \Delta f (x_0, x_1) + \cdots +
 (x - x_0) \cdots (x - x_{n-1}) \Delta^n f (x_0, \ldots x_n) + R\]
where the remainder can be expressed either as
 \[R = (x - x_0) \cdots (x - x_n) \Delta^{n+1} f (x, x_1, \ldots, x_n)\]
or as
 \[R = {1 \over (n+1)!} (x - x_0) \cdots (x - x_n) f^{(n+1)} (\eta)\]
where $\eta$ lies between the smallest and the largest of $x, x_0, 
\ldots, x_n$.

\textbf{Remark}.  If $f$ is a polynomial of degree $n$, then $R$ vanishes.
%%%%%
%%%%%
\end{document}
