\documentclass[12pt]{article}
\usepackage{pmmeta}
\pmcanonicalname{DividedDifferencesOfPowers}
\pmcreated{2013-03-22 16:47:59}
\pmmodified{2013-03-22 16:47:59}
\pmowner{rspuzio}{6075}
\pmmodifier{rspuzio}{6075}
\pmtitle{divided differences of powers}
\pmrecord{11}{39033}
\pmprivacy{1}
\pmauthor{rspuzio}{6075}
\pmtype{Theorem}
\pmcomment{trigger rebuild}
\pmclassification{msc}{39A70}

\endmetadata

% this is the default PlanetMath preamble.  as your knowledge
% of TeX increases, you will probably want to edit this, but
% it should be fine as is for beginners.

% almost certainly you want these
\usepackage{amssymb}
\usepackage{amsmath}
\usepackage{amsfonts}

% used for TeXing text within eps files
%\usepackage{psfrag}
% need this for including graphics (\includegraphics)
%\usepackage{graphicx}
% for neatly defining theorems and propositions
\usepackage{amsthm}
% making logically defined graphics
%%%\usepackage{xypic}

% there are many more packages, add them here as you need them

% define commands here
\newtheorem{thm}{Theorem}
\begin{document}
In this entry, we will prove the claims about divided differences 
of polynomials.  Because the divided difference is a linear
operator, we can focus our attention on powers.  

\begin{thm}
If $f(x) = x^n$ and $m \le n$, then
\[
\Delta^m f [x_0, \cdots x_m] = 
\sum_{k_0 + \cdots + k_m = n - m} x_0^{k_0} \cdots x_m^{k_m}.
\]
If $m > n$, then $\Delta^m f [x_0, \cdots x_m] = 0$.
\end{thm}

\begin{proof}
We proceed by induction.  The formula is trivially true
when $m = 0$.  Assume that the formula is true for a
certain value of $m$.  Then we have
\begin{align*}
\Delta^{m+1} f [x_0, \cdots x_{m+1}] &= 
{\Delta^m f [x_1, x_2 \cdots x_m] - \Delta^m f [x_0, x_2 \cdots x_m]
 \over x_1 - x_0} \\ &=
\sum_{k + k_2 + \cdots + k_m = n - m} 
{(x_1^k - x_0^k) x_2^{k_2} \cdots x_m^{k_m}
 \over x_1 - x_0}
\end{align*}
Using the identity for the sum of a geometric series,
\[
{x_1^k - x_0^k \over x_1 - x_0} =
\sum_{k_0 + k_1 = k - 1} x_0^{k_0} x_1^{k_1},
\]
this becomes
\begin{align*}
\Delta^{m+1} f [x_0, \cdots x_{m+1}] &= 
\sum_{k + k_2 + \cdots + k_m = n - m} \quad
\sum_{k_0 + k_1 = k - 1} 
x_0^{k_0} x_1^{k_1} x_2^{k_2} \cdots x_{m+1}^{k_{m+1}} \\ &=
\sum_{k_0 + k_1 + k_2 + \cdots + k_m = n - (m + 1)}
x_0^{k_0} x_1^{k_1} x_2^{k_2} \cdots x_{m+1}^{k_{m+1}}.
\end{align*}
Note that when $k=0$, we have $x_1^k - x_0^k = 0$, which
is consistent with the formula given above because, in that
case, there are no solutions to $k_1 + k_2 = k$, so the 
sum is empty and, by convention, equals zero.  Likewise,
when $n=m$, then the only solution to $k_0 + \cdots + k_m = 0$
is $k_0 = \cdots = k_m = 0$, so the sum only consists of
one term, $x_0^0 \cdots x_m^0 = 1$ so 
$\Delta^n f [x_0, \cdots x_n] = 1$, hence taking
further differences produces zero.
\end{proof}
%%%%%
%%%%%
\end{document}
