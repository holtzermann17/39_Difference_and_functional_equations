\documentclass[12pt]{article}
\usepackage{pmmeta}
\pmcanonicalname{IndefiniteSum}
\pmcreated{2013-03-22 17:35:14}
\pmmodified{2013-03-22 17:35:14}
\pmowner{CWoo}{3771}
\pmmodifier{CWoo}{3771}
\pmtitle{indefinite sum}
\pmrecord{20}{39999}
\pmprivacy{1}
\pmauthor{CWoo}{3771}
\pmtype{Definition}
\pmcomment{trigger rebuild}
\pmclassification{msc}{39A99}
\pmrelated{FiniteDifference}

\usepackage{amssymb,amscd}
\usepackage{amsmath}
\usepackage{amsfonts}
\usepackage{mathrsfs}
\usepackage{tabls}

% used for TeXing text within eps files
%\usepackage{psfrag}
% need this for including graphics (\includegraphics)
%\usepackage{graphicx}
% for neatly defining theorems and propositions
\usepackage{amsthm}
% making logically defined graphics
%%\usepackage{xypic}
\usepackage{pst-plot}
\usepackage{psfrag}

% define commands here
\newtheorem{prop}{Proposition}
\newtheorem{thm}{Theorem}
\newtheorem{ex}{Example}
\newcommand{\real}{\mathbb{R}}
\newcommand{\pdiff}[2]{\frac{\partial #1}{\partial #2}}
\newcommand{\mpdiff}[3]{\frac{\partial^#1 #2}{\partial #3^#1}}
\begin{document}
Recall that the finite difference operator $\Delta$ defined on the set of functions $\mathbb{R}\to \mathbb{R}$ is given by $$\Delta f(x):= f(x+1)-f(x).$$
The difference operator can be thought of as the discrete version of the derivative operator sending a function to its derivative (if it exists).  With the derivative operation, there corresponds an inverse operation called the antiderivative, which, given a function $f$, finds its antiderivative $F$ so that the derivative of $F$ gives $f$.  There is also a discrete analog of this inverse operation, and it is called the \emph{indefinite sum}.

The \emph{indefinite sum} of a function $f:\mathbb{R}\to \mathbb{R}$ is the set of functions $$\lbrace F:\mathbb{R}\to \mathbb{R}\mid \Delta F = f \rbrace.$$
This set is often denoted by $\Delta^{-1}f$ or $\Sigma f$, and any element in $\Delta^{-1}f$ is called an indefinite sum of $f$.

\textbf{Remark}.  Like the indefinite integral, the indefinite sum $\Delta^{-1}$ is shift invariant.  This means that for any $F\in \Delta^{-1} f$, then $F+c \in \Delta^{-1} f$ for any $c\in \mathbb{R}$.  But, unlike the indefinite integral, the indefinite sum is also invariant by a shift of a periodic real function of period $1$.  Conversely, the difference of two indefinite sums of a function $f$ is a periodic real function of period $1$.

In the following discussion, we consider the indefinite sum of a function as a function.

\textbf{Basic Properties}
\begin{enumerate}
\item $\Delta \Delta^{-1}f =f$, and $\Delta^{-1} \Delta f=f$ modulo a real function of period $1$.
\item Modulo a real number, and treating $\Delta^{-1}$ as an operator taking a function into a function, we see that $\Delta^{-1}$ is linear, that is, 
\begin{itemize}
\item $\Delta^{-1} (rf) = r\Delta^{-1} f$ for any $r\in \mathbb{R}$, and 
\item $\Delta^{-1}(f+g)=\Delta^{-1}f +\Delta^{-1}g$.
\end{itemize}
\item If $F(x)=\Delta^{-1}f(x)$, then $F(x+a)=\Delta^{-1}f(x+a)$.
\item If $F=\Delta^{-1}f$, then we see that 
\begin{eqnarray*}
F(a+1)-F(a) &=& f(a), \\ F(a+2)-F(a+1)&=& f(a+1), \\ &\vdots& \\ F(x)-F(x-1)&=& f(x-1).  
\end{eqnarray*}
where $x-a$ is a positive integer.  Summing these expressions, we get $$F(x)-F(a)=\sum_{i=1}^{x-a} f(a+i-1).$$  This is the discrete version of the fundamental theorem of calculus.
\end{enumerate}

Below is a table of some basic functions and their indefinite sums ($C$ is a real-valued periodic function with period $1$):

\begin{center}
\begin{tabular}{|c|c|c|}
\hline
$f(x)$ & $\Delta^{-1}f(x)$ & Comment \\
\hline\hline
$r\in \mathbb{R}$ & $rx+C$ & \\
\hline
$x$ & $\displaystyle{\frac{x(x-1)}{2}+C}$ & \\
\hline
$x^2$ & $\displaystyle{\frac{x(x-1)(2x-1)}{6}+C}$ & \\
\hline
$x^3$ & $\displaystyle{\frac{x^2(x-1)^2}{4}+C}$ & \\
\hline
$x^n$ & $T_n(x)+C$ & See this \PMlinkname{link}{SumOfPowers} for detail \\
\hline
$a^x$ & $\displaystyle{\frac{a^x}{a-1}+C}$ & $a\ne 1$ \\
\hline
$(x)_n$ & $\displaystyle{\frac{(x)_n}{n+1}+C}$ & $(x)_n$ is the falling factorial of degree $n$ \\
\hline
$\displaystyle{\binom{x}{n}}$ & $\displaystyle{\binom{x}{n+1}+C}$ & $\displaystyle{\binom{x}{n}:=\frac{(x)_n}{n!}}$ \\
\hline
$\displaystyle{\frac{1}{x}}$ & $\psi(x)+C$ & $\psi(x)$ is the digamma function \\
\hline
$\ln{x}$ & $\ln{\Gamma(x)}+C$ & $\Gamma(x)$ is the gamma function \\
\hline
$\sin{x}$ & $\displaystyle{-\frac{\cos(x-1/2)}{2\sin(1/2)}+C}$ & \\
\hline
$\cos{x}$ & $\displaystyle{\frac{\sin(x-1/2)}{2\sin(1/2)}+C}$ & \\
\hline
\end{tabular}
\end{center}

\begin{thebibliography}{9}
\bibitem{cj} C. Jordan. \emph{Calculus of Finite Differences}, third edition. Chelsea, New York (1965)
\end{thebibliography}
%%%%%
%%%%%
\end{document}
